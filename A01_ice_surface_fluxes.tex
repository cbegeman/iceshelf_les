The turbulence closure scheme is Anisotropic Minimum Dissipation (Rozema) employing the extension to scalars introduced by Abkar and Moin. Our implementation shows similar qualitative behavior to the stable atmospheric boundary layer published in XX; some inter-model variability is expected due to different numerics.

Momentum and scalar fluxes at the top of the domain are parameterized assuming law of the wall and a linear stability function in which depth is scaled by the Monin-Obukhov length, following Vreugdenhil and Taylor (2019). Details can be found in Appendix XX. Scalar fluxes at the top of the domain take the ``virtual" freshwater flux form \cite{XX}. The solution for ice-shelf melting at the interface follows the standard procedure of solving three equations -- local heat conservation, local salt conservation, and the interface temperature set to the freezing point -- for three unknowns, the temperature and salinity at the ice-ocean interface, and the melt rate. This is the so-called three-equation parameterization. For the freezing temperature equation we use the polynomial function from \cite{jackett_2006}. We also assume that the conductive heat flux into ice is zero, as it is typically no more than 10\% of the latent heat used for melting.