\documentclass[letterpaper,10pt]{report}

\usepackage{ifxetex,ifluatex} % provides way to check if document is being processed with xetex or luatex

\usepackage[colorlinks = true,
unicode=true,
linkcolor = Black,
urlcolor  = Black,
citecolor = Black,
anchorcolor = blue]{hyperref}

\usepackage[english]{babel} % language support

% --- page layout ---

\usepackage{lastpage} % allows user to refer to last page with label LastPage
\usepackage{titlesec} % options for title, headers, contents

%header
\usepackage{fancyhdr} % customize headers and footers
\pagestyle{fancy}
\fancyhf{}
\rhead{Carolyn Begeman, PALM development, Page \thepage \thinspace \thinspace of \pageref{LastPage}}
\lhead{}
\renewcommand{\headrulewidth}{0pt}

% page dimensions
\usepackage{geometry}
\geometry{letterpaper, top=0.75in, left=1in, total={6.5in,9.5in}, headheight=12pt, headsep=.1in} % topmargin=-23pt, 

\makeatletter
\def\maxwidth{\ifdim\Gin@nat@width>\linewidth\linewidth\else\Gin@nat@width\fi}
\def\maxheight{\ifdim\Gin@nat@height>\textheight\textheight\else\Gin@nat@height\fi}
\makeatother

% set paragraph settings
\IfFileExists{parskip.sty}{%
	\usepackage{parskip}
}{% else
	\setlength{\parindent}{0pt}
	\setlength{\parskip}{6pt plus 2pt minus 1pt}
}
\setlength{\emergencystretch}{3em}  % prevent overfull lines
\providecommand{\tightitemize}{%
	\setlength{\itemsep}{0pt}\setlength{\parskip}{0pt}}

\setcounter{secnumdepth}{3} % set maximum depth of subsections

%\usepackage{caption}
%\captionsetup[table]{skip=8pt]}

% --- font-related packages ---

\usepackage{mathptmx} % make Times default font, with math font
\usepackage{lmodern} % load Latin Modern family of fonts
\usepackage{amssymb,amsmath} % define AMS symbol fonts
\usepackage[usenames,dvipsnames]{color} % color model

% use upquote if available, for straight quotes in verbatim environments
\IfFileExists{upquote.sty}{\usepackage{upquote}}{}

% use microtype if available
\IfFileExists{microtype.sty}{%
	\usepackage[]{microtype}
	\UseMicrotypeSet[protrusion]{basicmath} % disable protrusion for tt fonts
}{}

\PassOptionsToPackage{hyphens}{url} % url is loaded by hyperref
\hypersetup{
	pdfborder={0 0 0},
	breaklinks=true}
%\urlstyle{same}  % don't use monospace font for urls
%\usepackage[utf8]{inputenc} % specify input encoding, \DeclareUnicodeCharacter

\ifnum 0\ifxetex 1\fi\ifluatex 1\fi=0 % if pdftex
\usepackage[T1]{fontenc}
\usepackage[utf8]{inputenc}
\else % if luatex or xelatex
\ifxetex
\usepackage{mathspec}
\else
\usepackage{fontspec}
\fi
\defaultfontfeatures{Ligatures=TeX,Scale=MatchLowercase}
\fi

% set fontsize for section and subsection
\newcommand{\secfnt}{\fontsize{12}{14}}
\titlespacing\section{0pt}{0pt plus 2pt minus 2pt}{0pt plus 2pt minus 2pt}
\newcommand{\ssecfnt}{\fontsize{12}{14}}
\titlespacing\subsection{0pt}{0pt plus 2pt minus 2pt}{0pt plus 2pt minus 2pt}
\renewcommand{\baselinestretch}{1}

% settings for text that designates code
\definecolor{light-gray}{gray}{0.95} 
\newcommand{\code}[1]{\texttt{#1}} %{\colorbox{light-gray}

% --- itemizes ---
\usepackage{enumitem,amssymb}
\newitemize{todoitemize}{itemize}{2}
\setitemize[todoitemize]{label=$\square$}
\usepackage{pifont}
\newcommand{\cmark}{\ding{51}}%
\newcommand{\xmark}{\ding{55}}%
\newcommand{\done}{\rlap{$\square$}{\raisebox{2pt}{\large\hspace{1pt}\cmark}}%
	\hspace{-2.5pt}}
\newcommand{\wontfix}{\rlap{$\square$}{\large\hspace{1pt}\xmark}}

% --- graphics ---

\usepackage{graphicx,grffile}
\usepackage{makecell}

% Scale images if necessary, so that they will not overflow the page
% margins by default, and it is still possible to overwrite the defaults
% using explicit options in \includegraphics[width, height, ...]{}
\setkeys{Gin}{width=\maxwidth,height=\maxheight,keepaspectratio}

% set default figure placement to htbp
\makeatletter
\def\fps@figure{htbp}
\makeatother

%\graphicspath{ {} }

%\date{} %replace date with text \date{text}

\begin{document}%\thispagestyle{empty}
\tableofcontents{}
\chapter{Motivation}

% from CK application
Small scale, turbulent flow below ice shelves is regionally isolated and difficult to measure and simulate.  Yet these small scale processes, which regulate heat transfer between the ocean and ice shelves, can have global-scale climate impacts; small increases in ice shelf melting decrease the ability of Antarctic ice shelves to “buttress” ice flux to the ocean, potentially allowing for runaway ice loss and sea-level rise.  These critical processes are not captured by even the most advanced Earth System Models (ESMs), such as the DOE’s Energy Exascale Earth System Model (E3SM), of which LANL is a major developer. Direct measurements of ocean properties below ice shelves are exceedingly rare, expensive to perform, and difficult to generalize to larger areas. Further, the few available measurements indicate that existing boundary layer theory on sub-ice-shelf turbulence (based largely on observations below sea ice) is incomplete. A fundamental understanding of turbulent flow in this unusual regime is lacking, despite widespread adoption and application of existing parameterizations by the international ocean, ice sheet, and ESM communities.

% from postdoc proposal
The largest source of uncertainty in future sea level rise is the potential loss of ice from the Antarctic Ice Sheet. The rate of grounded ice loss is highly sensitive to the melting of ice shelves, which drain over 80\% of Antarctica’s grounded ice. In turn, the Ice-Ocean Boundary Layer (IOBL) controls ice-shelf melting by regulating oceanic heat and salt fluxes to the ice shelf base; accurate predictions of ice-shelf melting depend on capturing the turbulent dynamics of the IOBL. One indication that ocean models do not capture these dynamics is that the modeled thickness of the IOBL differs significantly between ocean models and with model resolution. Furthermore, ocean models predict ice-shelf melting using parameterizations that neglect the buoyancy of the IOBL. This model deficiency likely biases turbulent fluxes through the IOBL and sub-ice-shelf ocean circulation, which is primarily driven by the buoyant flow of water freshened by ice-shelf melting. A new parameterization of ice-shelf melting that accounts for boundary layer dynamics is needed to achieve a more physically-based, accurate coupling of ice sheets and oceans in climate models.

% plan
I propose to develop a new parameterization of ice-shelf melting by modeling turbulent heat, salt, and momentum fluxes through the IOBL using Large-Eddy Simulation (LES). Whereas ocean models typically used to model sub-ice-shelf ocean cavities cannot capture the relevant turbulent scales for boundary layer dynamics, LES captures the dominant energy-containing scales of turbulence and represents smaller, unresolved scales with varying degrees of complexity. During the first year of the project, I will simulate the IOBL using the MIT General Circulation Model (MITgcm) in a LES ocean configuration. An effective parameterization of ice-shelf melting is likely to depend on ice-shelf slope, the temperature, salinity, and velocity outside the IOBL, and the depth of the ice-ocean interface. I will vary these parameters between model runs and characterize turbulent fluxes, IOBL thickness, and ice-shelf melt rates when the modeled IOBL has equilibrated.  During the second year of the project, I will implement the resulting parameterization in the Model for Prediction Across Scales Ocean (MPAS-O), the ocean component of DOE’s new Energy Exascale Earth System Model (E3SM). Using MPAS-O in idealized configurations, I will investigate the parameterization’s impact on sub-ice-shelf ocean circulation and the distribution of ice-shelf melting as compared with previous parameterizations. I will focus on the sensitivity of ice-shelf melting to increasing seawater temperature, a trend observed along a wide swath of the West Antarctic coastline and a potential trigger for West Antarctic Ice Sheet collapse. Resources for this work include MPAS-O, developed at LANL, and existing allocations at a number of DOE HPC facilities (including LANL). 

The proposed work, characterizing turbulence in the presence of a buoyancy flux and sloping boundary, addresses a fundamental problem in fluid dynamics. The work also has direct societal relevance because ice-ocean interactions are believed to be a major control on future sea level rise. This new parameterization for ice-shelf melting will support an ongoing effort at LANL and other national labs to accurately couple ice sheet and ocean models and will be exportable from MPAS-O to other models. Thus, in addition to significantly improving simulations of ocean circulation, Antarctic ice loss, and sea level rise in E3SM, it could provide similar advances to other Earth system modeling efforts. 

%\chapter{Relevant literature}
%Put something here

\chapter{The PALM model}
    \section{Table of variables}
    
    \textbf{Acronyms}:
    SGS = sub-grid-scale
    TKE = turbulent kinetic energy
    
    \begin{tabular}{c|c}
       $e$      & SGS TKE \\
       $\theta$ & potential temperature \\
       $S$      & practical salinity
    \end{tabular}
	\section{Model overview}
	
	The PArallelized Large eddy simulation Model (PALM) was developed at the Institute of Meteorology and Climatology at Leibniz Universitat Hannover (Germany). It has been applied to the simulation of atmospheric and ocean boundary layers. However, it has never been used specifically to treat a sub-ice-shelf ocean boundary layer. For this application, the key model development was the implementation of dynamic melting fluxes. 
	
	Prognostic equations:
	Momentum conservation
	\begin{equation} \label{eq:uprog}
	\frac{\partial u_i}{\partial t} = 
	-\frac{\partial u_i u_j}{\partial x_i}
	-\varepsilon_{ijk} f_j u_k 
	+ \varepsilon_{i3j} f_3 u_{g,j} 
	- \frac{1}{\rho_0}\frac{\partial \pi^*}{\partial x_i} 
	+ g_3 \frac{\rho - \langle \rho \rangle}{\langle \rho \rangle}\delta_{i3} 
	+ g_1 \frac{\rho - \langle \rho \rangle}{\langle \rho \rangle}\delta_{i3} 
	- \frac{\partial}{\partial x_j}(\overline{u''_i u''_j} - \frac{2}{3}e\delta_{ij})
	\end{equation}
	
	where \\
	$g = g [sin \alpha,0,cos \alpha]$\\
	$f = 2 \Omega [sin \phi sin \alpha,cos \phi,sin \phi cos \alpha]$
	
	Mass conservation for incompressible flows
	\begin{equation} \label{eq:volconserv}
	\frac{\partial u_j}{\partial x_j} = 0
	\end{equation}
	
	Heat and salt conservation
	\begin{equation} \label{eq:ptprog}
	\frac{\partial \theta}{\partial t} = -\frac{\partial u_j \theta}{\partial x_j} - \frac{\partial}{\partial x_j}(\overline{u''_i \theta''} - F_T)
	\end{equation}
	
	\begin{equation} \label{eq:saprog}
	\frac{\partial S}{\partial t} = -\frac{\partial u_j S}{\partial x_j} - \frac{\partial}{\partial x_j}(\overline{u''_i Sa''} + F_S)
	\end{equation}
	
	Turbulence closure by the gradient mean approximation
	\begin{equation} \label{eq:momdiff}
	\overline{u''_i u_j''} - \frac{2}{3}e \delta_{ij} = -K_m(\frac{\partial u_i}{\partial x_j} + \frac{\partial u_j}{\partial x_i})
	\end{equation}
	
	\begin{equation} \label{eq:ptdiff}
	\overline{u''_i \theta''} = -K_h \frac{\partial \theta}{\partial x_i}
	\end{equation}
	
%	Prognostic equation for Resolved TKE
%	\begin{equation} \label{eq:eprog}
%	\frac{\partial e}{\partial t} = 
%	-u_j \frac{\partial e}{\partial x_j} 
%	- (\overline{u^*_i u^*_j})\frac{\partial u_i}{\partial x_j} + \frac{g_i}{\rho_0}\overline{u^*_i \rho^*}
%	\end{equation}
	
	Prognostic equation for SGS TKE
	\begin{equation} \label{eq:eprog}
	\frac{\partial e}{\partial t} = 
	-u_j \frac{\partial e}{\partial x_j} 
	- (\overline{u''_i u''_j})\frac{\partial u_i}{\partial x_j}
	+ \frac{g_i}{\rho_0}\overline{u''_i \rho''}
	- \frac{\partial}{\partial x_j}[\overline{u_j''(e+\frac{p''}{\rho_0})}]
	- \varepsilon
	\end{equation}
	
	modified perturbation pressure:
	\begin{equation} \label{eq:pi}
	\pi^* = p^* + \frac{2}{3}\rho_0 e
	\end{equation}
	where $p^*$ is the perturbation pressure and the turbulent kinetic energy $e$ is

	\begin{equation} \label{eq:e}
	e = \frac{1}{2}\overline{u''_i u''_i}
	\end{equation}
	
	\begin{equation} \label{eq:sadiff}
	\overline{u''_i Sa''} = -K_h \frac{\partial Sa}{\partial x_i}
	\end{equation}
	
	Eddy diffusivity of momentum
	\begin{equation} \label{eq:Km}
	K_m = c_m l \sqrt{e}
	\end{equation}

	Eddy diffusivity of heat and salt
	\begin{equation} \label{eq:Kh}
	K_h = (1+\frac{2l}{\Delta})K_m
	\end{equation}
    where
	\begin{equation} \label{eq:gridl}
	\Delta = \sqrt[3]{\Delta x \Delta y \Delta z}
	\end{equation}
	
	%The model enforces incompressibility with 
	%\begin{equation} \label{eq:pdiv}
	%\frac{\partial^2 \pi^{*t}}{\partial x_i^2} = \frac{\rho_0}{\Delta t} \frac{\partial u_{i,pre}^{t + \Delta t}}{\partial x_i}
	%\end{equation}
	\newpage
	\subsection{Sloping surface}
	We represent a sloping ice base by rotating the gravity vector while keeping the domain a rectangular prism. The ice base always slopes in the x-dimension of our domain. The sloping surface modifies buoyancy terms in the model and imposes a surface pressure gradient. The surface pressure gradient was accurate in PALM v0, but the buoyancy terms were based only on temperature rather than density. We define buoyancy based on the in-situ density, $g cos(\alpha) \rho/\rho_0$.
		
	Sloping surface requires cyclic boundary conditions. There are two possible approaches in the sloping surface case:
	(a) Slope offset: Choose initial conditions in which isopycnals follow isobars. When water is cycled across boundaries, apply a temperature offset determined by the slope and the initial temperature gradient. Apply a salinity offset as well (not implemented in PALM v0).
	(b) No offset: Choose initial conditions in which isopycnals are parallel to the surface. Do not apply any temperature or salinity offset to water cycled across boundaries.  
	
	\textbf{Development}:
	PALM vslope: Analogous to slope offset in temperature field, define slope offset for salinity field.
	\begin{itemize}
		\item To do itemize
		\begin{todoitemize}
			\item Add \code{sa\_slope\_offest} to \code{advec\_s\_bc}
			\item Option to turn slope offset off, would make \code{slope\_offset}=0 in \code{init\_slope}
			\item Option to set initial temperature and salinity gradients parallel to ice base. Wouldn't need to recompute \code{pt\_init} or \code{sa\_init} in \code{init\_slope} (Same as last item?).
		\end{todoitemize}
	\end{itemize}
	\newpage		
	\section{Monin-Obukhov Similarity Theory for the surface layer}
    McPhee (1983) presented a shape function for stress as a function of distance from the surface, which we use here to solve for momentum fluxes near the surface. 
    
    First, we choose a depth at which the horizontal velocity is thought to be representative of the boundary layer flow (i.e., at least a couple grid points away from the interface where boundary layer effects can be unreaitemizeic). We denote this depth $z_m$. 
    
    The friction velocity is defined following "law of the wall"
    \begin{equation}
        u_* = \kappa u_m / log(z_m/z_0)
    \end{equation}
    
    The shape profile for stress is defined as a function of non-dimensional depth $\zeta$
    \begin{equation}
        \zeta = \frac{f z}{\eta_* u_*}
    \end{equation}
    
    $\eta_*$ is a scaling parameter meant to account for the stabilizing effects of a buoyancy flux, equal to 1 when melt rates are 0 and approaching 0 when melt rates are high.
    \begin{equation}
        \eta_*^2 = (1+\frac{\xi_N u_*}{f L_O Ri_c})^{-1}
    \end{equation}
    where $L_O$ is the Monin-Obukhov length
    \begin{equation}
        L_O = \rho \frac{u_*^3}{\kappa F_{B}}
    \end{equation}
    
    The momentum flux, expressed in imaginary notation where real values are aligned with the x-axis and imaginary values are aligned with the y-axis:
    \begin{equation}
        \tau = exp(\frac{i \zeta}{\kappa \xi_N})^{1/2}
    \end{equation}
    \begin{equation}
        u'w' = \rho |u_*| (u_* RE(\tau) + v_* IM(\tau))
    \end{equation}
    \begin{equation}
        v'w' = \rho |u_*|(v_* RE(\tau) + u_* IM(\tau))
    \end{equation}
    \newpage
    \section{Nondimensionalization}
    
    \begin{equation}
        Z = \frac{z}{h}
    \end{equation}
    Define domain size in terms of $Z$
    
    \begin{equation}
        h_{\nu} = u_*/\nu
    \end{equation}
    
    \begin{equation}
        h_{BL} = u_* \eta_*/f
    \end{equation}
    
    \begin{equation}
        U = \frac{u}{u_*}
    \end{equation}
    
    \begin{equation}
        U = \frac{u}{u_g}
    \end{equation}
    
    \begin{equation}
        u_* = (\frac{1}{\rho}\int_0^h \frac{\partial \overline{p}}{\partial x} dz)^{1/2} 
    \end{equation}
    
    \begin{equation}
        u_g = - \frac{dp/dy}{\rho f}
    \end{equation}

    \begin{equation}
        \Theta = \frac{\theta - \theta_\infty}{Q_\theta/u_*}
    \end{equation}
    $(\theta - \theta_\infty)$ and $Q_\theta$ are derived from output. $Q_x$ is the scalar flux at the interface
    
    \begin{equation}
        Q_x = \kappa_x \frac{\partial X}{\partial z}
    \end{equation}

    \newpage
    
    \section{Melt rate solution method}
	% gamma_X
	% Depends on positive melt rate and that the effect of stratification is to reduce fluxes rather than enhance fluxes as a result of buoyancy from sloping surface
	
	Asay-Davis Eq. 24-26
	\begin{equation}
	\rho_{sw} \gamma_T (T_b-T) = \rho_{fw} m(\frac{-L_i}{c_p})
	\end{equation}
	\begin{equation}
	\rho_{sw} \gamma_S (S_b-S) = \rho_{fw} m S_b
	\end{equation}
	\begin{equation}
	T_b = T_b^{n-1}+c_2 S_b - c_2 S_b^{n-1}
	\end{equation}
	
	\begin{equation}
	(\gamma_T c_2) S_b^2 + (\gamma_T(T_b^{n-1}-c_2 S_b^{n-1}-T)-\gamma_S \frac{-L_i}{c_p})S_b + \gamma_S\frac{-L_i}{c_p}S=0
	\end{equation}
	
	
	%Asay-Davis Eq. 29-30
	%\begin{equation}
	%	F_T = -c_p(\rho_{sw} \gamma_T + \rho_{fw} m)(T_b-T)
	%\end{equation}
	%$F_T > 0, m > 0, T_b-T > 0$
	
	%\begin{equation}
	%	F_S = -(\rho_{sw} \gamma_S + \rho_{fw} m)(S_b-S)
	%\end{equation}
	%$F_S > 0, m > 0, S_b-S < 0$
	
	%Substituing the melt rate equation, these equations can also be expressed as
	%\begin{equation}
	%F_T = -\rho_{sw} c_p(\gamma_T - \gamma_S \frac{S_b-S}{S_b})(T_b-T)
	%\end{equation}
	%\begin{equation}
	%F_S = -\rho_{sw} (\gamma_S - \gamma_S \frac{S_b-S}{S_b})(S_b-S)
	%\end{equation}
	
	\begin{equation}
	F_B = g(\beta F_S - \alpha F_T)
	\end{equation}
	
	%$F_T,F_S$ stored in \code{surf\_def\_h}.
	
	General Implementation strategy
	\begin{enumerate}
		\item Initialization: 
		a. shf, sasws are assigned (to 0), 
		b. Friction velocity is initialized (to 0?). 
		c. Define buoyancy flux from shf and sasws 
			(need to have alpha and beta at the surface)
		\item Calculate exchange velocities $\gamma_T,\gamma_S$ as described in the previous section based on shf, sasws, us at previous timestep
		\item Use 3 equations to solve for $S_b, T_b$
		\item Iterate if $S_b$ is significantly different from prior $S_b$ used to define $dT_f/dS$.
		\item Calculate heat, salt fluxes from 3 equations
		\item Iterate 2-5 if buoyancy flux is significantly different than initial buoyancy flux
		% consider calculating momentum fluxes
	\end{enumerate}

	\begin{table}[htbp]%{\textwidth}
	\caption{$\Delta z = $vertical grid dimension, $\gamma_{fr} = $exchange velocity for freezing case, $z_m = $distance from the interface for boundary layer conditions, $\Delta z_m = $averaging width for boundary layer conditions}
	\label{table:most_prm}
    \centering
    \begin{tabular}{|l|l|l|}
	\hline
	Parameter & Optimal value & Sensitivity test outcome\\
	\hline
	$L_O$ minimum & $\Delta z/20$ & \makecell[l]{X\% of the domain reaches $L_O$ minimum \\ for X temperature case} \\ \hline
	$L_O$ maximum & $20\Delta z$ & \makecell[l]{X\% of the domain reaches $L_O$ maximum \\ for X temperature case}\\ \hline
    $\gamma_{fr}$ & $(5.7e-3) u_*$ & \makecell[l]{X\% of the domain is freezing \\ for X temperature case}\\ \hline
    $z_m$ & max(z(min $d\theta/dz$),z(min $dS/dz$)) & not tested\\ \hline %up to 2 x boundary layer depth
    $\Delta z_m$ & $0 - 5$ & not tested\\ \hline
    $\Delta x_m = \Delta y_m$ & $\Delta z_m, L_x$ & not tested\\ \hline
	$c_d$ & $(1e-3) - (5e-3)$ & not tested\\ \hline
    \end{tabular}
	\end{table}
	
	%can argue that $L_O$ minimum and $\gamma_fr$ don't matter if for the cases we run so little of the domain is freezing that it doesn't modify mean fluxes of momentum, heat and salt
	\newpage
	\section{Initial and boundary conditions}
	
	Velocity:
	\begin{itemize}
	\item Cyclic horizontal boundary conditions
	\item \textbf{u} = 0 at the ice-ocean interface
	\item Bottom boundary: u,v consistent with horizontal pressure gradients. w=0
	\end{itemize}
	
	Pressure:
	\begin{itemize}
	\item Prescribed horizontal pressure gradients
	\item Hydrostatic initial pressure based on prescribed surface pressure and initial density profile.
	\end{itemize}
	
	Temperature:
	\begin{itemize}
	    \item Cyclic horizontal boundary conditions
	    \item far-field temperature relaxed at the bottom third of the domain
	    \item interface temperature determined by melt rate parameterization, set to the freezing point
	\end{itemize}
	
	Salinity:
    \begin{itemize}
	    \item Cyclic horizontal boundary conditions
	    \item far-field salinity relaxed at the bottom third of the domain
	    \item interface salinity determined by melt rate parameterization
	\end{itemize}
	
	\newpage
	
	\section{Parameter space to explore}
	

diagnostics:

anisotropy measures

velocity anisotropy tensor (symmetric, traceless, no elements in isotropic flow):
\begin{equation}
	b_{i,j} = \frac{<u_i'u_j'>}{<u_k'u_k'>} - \frac{\delta_ij}{3}
\end{equation}
can be expressed similarly for the scalar gradient field (if you have the fluctuations along x,y,z axes)
and the vorticity field where $\omega = \grad x u$

\begin{equation}
b_{11} = \frac{<u_1'u_1'>}{<u_2'u_2'>} 
+ \frac{<u_1'u_1'>}{<u_3'u_3'>} 
\end{equation}
\begin{equation}
	b_{12} = \frac{<u_1'u_2'>}{<u_1'u_1'>} 
	              + \frac{<u_1'u_2'>}{<u_2'u_2'>} 
	              + \frac{<u_1'u_2'>}{<u_3'u_3'>}
\end{equation}
\begin{equation}
b_{22} =  \frac{<u_1'u_1'>}{<u_2'u_2'>} 
+ \frac{<u_1'u_1'>}{<u_3'u_3'>}
\end{equation}
\begin{equation}
b_{23} = \frac{<u_2'u_3'>}{<u_1'u_1'>} 
+ \frac{<u_2'u_3'>}{<u_2'u_2'>} 
+ \frac{<u_2'u_3'>}{<u_3'u_3'>}
\end{equation}
\begin{equation}
b_{13} = \frac{<u_1'u_3'>}{<u_1'u_1'>} 
+ \frac{<u_1'u_3'>}{<u_2'u_2'>} 
+ \frac{<u_1'u_3'>}{<u_3'u_3'>}
\end{equation}
\begin{equation}
b_{33} = \frac{<u_3'u_3'>}{<u_1'u_1'>} 
+ \frac{<u_3'u_3'>}{<u_2'u_2'>} 
\end{equation}
where angle brackets are a volume average

code up the anisotropy tensor:
b11 = u*2/v*2 + u*2/w*2
b22 = v*2/u*2 + v*2/w*2
b33 = w*2/u*2 + w*2/v*2
b12 = u*v*/u*2 + u*v*/v*2 + u*v*/w*2
b13 = u*w*/u*2 + u*w*/v*2 + u*w*/w*2
b22 = v*w*/u*2 + v*w*/v*2 + v*w*/w*2

look at the evolution of these through time
and look at the depth distribution

\end{document}
