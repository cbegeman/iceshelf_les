Momentum flux

Components of surface stress $\tau_i$
\begin{equation*}
\tau_i = \rho \overline{u_i''w''}
\end{equation*}
for i = 1,2.

Friction velocity $u_*$
\begin{equation*}
u_*^2=|\tau_0|^{1/2}
\end{equation*}
% \begin{equation*}
% u_*^2=(\overline{u''w''}^2 + \overline{v''w''}^2)^{1/2}
% \end{equation*}

Turbulence closure
\begin{equation*}
\overline{u_i''w''} = k_m \frac{\partial u_i}{\partial z}
\end{equation*}

There is empirical support for a linear increase in eddy diffusivity from the wall
\begin{equation*}
u_* = \frac{k z}{\Psi_m(\zeta) } \frac{\partial u}{\partial z}
\end{equation*}
where $\zeta = z/L_O$

We integrate using this relationship (or $\Psi = 1$ to get $u_*$
\begin{equation*}
\Psi_m(\zeta) = 1 + \beta_m \zeta
\end{equation*}
where $\beta_m = 4.8$ (Wyngaard 2010; Zhou et al. 2017)

We integrate using this relationship to get $u_i''w''$
\begin{equation*}
\beta_m = (\frac{1}{R_c} + \frac{f L_O}{u_* \chi_N})(1-\eta_*)
\end{equation*}
(McPhee 1981)

% Integrating for $u(z)$ (this is never needed because all we need is stress
% \begin{equation*}
% u = \frac{u_*}{k} ln(frac{z}{z_0} + \frac{\beta_m}{k}\zeta + C_m
%\end{equation*}
% \begin{equation*}
% u = \frac{u_*}{k} ln(frac{z}{z_0} = c_d(z)^{\frac{-1}{2}}u_*
% \end{equation*}
% The friction velocity is defined following "law of the wall"
% \begin{equation}
%     u_* = \kappa u_m / log(z_m/z_0)
% \end{equation}

% \begin{equation}
% \overline{u_i'w'} + = c_D^{1/2} u_i(z_1) u_*
% \overline{u_i'w'} = \frac{k z}{\Psi_m(\zeta) } \frac{\partial u_i}{\partial z}
% \end{equation}

% \begin{equation}
% \overline{u_i'w'} + = c_D^{1/2} u_i(z_1) u_*
% \overline{u_i'w'} = \frac{k z}{\Psi_m(\zeta) } \frac{\partial u_i}{\partial z}
% \end{equation}

\begin{equation*}
\overline{u_i''w''}(z_1) = u_* (\phi_i(\textbf{u(z_1)}))
\end{equation*}

McPhee (1983) presented a shape function for stress as a function of distance from the surface, which we use here to solve for momentum fluxes near the surface. 

First, we choose a depth at which the horizontal velocity is thought to be representative of the boundary layer flow (i.e., at least a couple grid points away from the interface where boundary layer effects can be unrealistic). We denote this depth $z_m$. 

The shape profile for stress is defined as a function of non-dimensional depth $\zeta$
\begin{equation}
    \zeta = \frac{z}{\eta_* u_*/f}
\end{equation}

$\eta_*$ is a scaling parameter meant to account for the stabilizing effects of a buoyancy flux, equal to 1 when melt rates are 0 and approaching 0 when melt rates are high.
\begin{equation}
    \eta_*^2 = (1+\frac{\xi_N u_*}{f L_O Ri_c})^{-1}
\end{equation}
where $L_O$ is the Monin-Obukhov length
\begin{equation}
    L_O = \rho \frac{u_*^3}{\kappa F_B}
\end{equation}

\begin{equation}
F_B = g(\beta w'S' - \alpha w'\theta')
\end{equation}

\begin{equation}
    \tau = exp(\frac{i \zeta}{\kappa \xi_N})^{1/2}
\end{equation}
\begin{equation}
    \phi_1 = (u_1* RE(\tau) + u_2* IM(\tau))
\end{equation}
\begin{equation}
    \phi_2 = (u_2* RE(\tau) + u_1* IM(\tau))
\end{equation}

\begin{equation}
    u''w'' = \rho |u_*| (u_* RE(\tau) + v_* IM(\tau))
\end{equation}
\begin{equation}
    v''w'' = \rho |u_*|(v_* RE(\tau) + u_* IM(\tau))
\end{equation}

% if not rotation
\begin{equation}
\textbf{\tau} = \rho \overline{\textbf{u}'w'} = \rho \frac{\textbf{u}}{|u|}u_*^2
%\overline{u_i'w'} + = c_D^{1/2} u_i(z_1) u_*
%\overline{u_i'w'} = \frac{k z}{\Psi_m(\zeta) } \frac{\partial u_i}{\partial z}
\end{equation}

\begin{equation}
%\frac{\partial u}{\partial z} = \frac{u_*}{k z}\Psi_m(\zeta) 
u_* = \frac{\kappa z}{\Psi_m(\zeta) } \frac{\partial |u|}{\partial z}
\end{equation}

\begin{equation}
u_* = \kappa u(z_1)(\textrm{ln}(z_1/z_0)-\Psi_m(\zeta_1))^{-1}
\end{equation}

